\documentclass[letter,scriptaddress,twocolumn, prl,showkeys]{revtex4-1}

    \usepackage{microtype}
    \usepackage{blindtext}

	\usepackage{amsmath}%,amssymb} 
	\usepackage{makeidx}
	\usepackage{amsfonts}
	\usepackage[ansinew]{inputenc}
	\usepackage[usenames,dvipsnames]{pstricks}
	\usepackage{subfigure}
	\usepackage{epsfig}
%	\usepackage{pst-grad} % For gradients
	\usepackage{pst-plot} % For axes
	\usepackage[colorlinks,hyperindex]{hyperref}
	\hypersetup
	{
		colorlinks,%
		citecolor=black,%
		linkcolor=black,%
		urlcolor=black,%
	}

%--- Theorem like environments ----
	\newtheorem{theorem}{Theorem}
	\newtheorem{corolary}{Corolary}
	\newtheorem{prop}{Proposition}
	\newtheorem{definition}{Definition}
	\newtheorem{example}{Example}
	\newtheorem{exercise}{Exercise}
	\newtheorem{lemma}{Lemma}

%--- Definindo algumas frescuras.
%\numberwithin{equation}{subsection}
	%\numberwithin{equation}{section}

	\setlength\textheight{24.5cm}
	
\makeindex

%--------------------------------------------------------
\begin{document}

\title{Optimizing Structural Properties Of Neural Networks With Genetic Algorithms}

\author{T. Staudt, E. Schultheis}
\email{thomas.staudt@stud.uni-goettingen.de, erik.schultheis@uni-goettingen.de}
\affiliation{University of G�ttingen}

\date{today}

\begin{abstract}
    In this article we examine how structural properties of neural
    networks, like the size or local synapse densities, affect their
    learning success for various tasks. To do so, we look at a rate based
    model for neural networks and apply the FORCE rule for the learning
    process. A sophisticated matching algorithm
    allows us to quantify the learning success of a network so that the
    fitness of structural characteristics, expressed by integer or
    floating-point parameters, can be evaluated. We then use evolutionary
    optimization methods on these parameters to show that (1) ring
    topologies are generally more successful than strictly random
    topologies, (2) *FAT
    LIST OF INCREDIBLE RESULTS THAT CHANGE THE UNIVERSE*
\end{abstract}

\keywords{genetic algorithms, evolutionary algorithms, FORCE learning, rate networks}

\maketitle

\section{Introduction}
In contemporary artificial neuronal network science a vast number of
different models for describing both the dynamic of networks as well as
their plasticities are applied. 
But what exactly makes artificial neuronal networks learn successfully? And
why are certain types of networks and certain learning rules effective for
some problems, but fail at solving others? Confronting and eventually
answering these questions is critical when trying to gain a deep
understanding of neural networks and artificial intelligence. 

Using the above questions as a guideline,
we decided to analyze the FORCE algorithm introduced in \cite{FORCE} that
is especially suitable for learning periodic patterns. More precisely, we
wanted to identify which properties of a network are important for its
learning success, and how to choose these properties' values for an optimal
result.  To this end, one option would be to take an initial network and
an adequate periodic pattern to be learned, and to optimize the network's
internals for this very task. Then patterns may emerge that hint at how an
optimal network design looks like. 

However, we took a different approach:
Instead of fine tuning single internal weights, we were concerned with the
general rules of the network's structure. So we wanted our networks to
be random to a certain degree, which seems biologically plausible, but we
also wanted them to follow certain patterns and allow for specialization:
How dense are the synapses distributed? Is there a strong compartmentalization?
Are recurrent structures frequent? How strong should the feedback signal
be, and how strong should the synapses fire on average?

To address these issues we looked at \emph{parameterized random networks}, or
in the language of genetics, at different \emph{network genotypes}. These
genotypes contain parametric information about the networks to be created,
the \emph{phenotypes}. Different phenotypes with the same genotype may thus
have strong quantitative differences, but they share structural qualities
whose fitness for a preassigned task we wanted to analyze and improve.

Since calculating the fitness of a genotype required the evaluation of many
different phenotypes and because of the resulting stochastic nature of the
problems we faced, we decided to use genetic algorithms for the
optimization process. Other more direct optimization methods like simulated
annealing or gradient descent seemed unfeasible for this.


\section{Concepts and Methods}
\paragraph{Basic Network Dynamics} As our neuronal network model we choose
basic rate networks that are also used in the original FORCE publication
\cite{FORCE} and the publication \cite{FORCE}, where our notation of this
paragraph mainly stems from. So we look at 

- describe network dynamic
- describe learning and tasks
- describe sucess of network

- We chose rate networks in order to apply FORCE rule
- borrow notation from reward modulation rule paper
Basic Network dynamics:
    - Introduce single neuronal network as collection of matrices with specified
    dynamics
    - Briefly describe the learning rule (FORCE)

\paragraph{Genes and Challenges}
    - Introduce parameterized random networks as generators of these networks
    - Introduce challenges, and the expected success values when paring
    generators with challenges.

\paragraph{Genetic Optimization}
Genetic Optimization:

- Hint / vague description of the codebase

\paragraph{Tasks}
For the optimization to work properly, the problems presented to the networks 
have (for the most part) to solvable by FORCE learning. While it was shown \cite{}
that large recurrent neural networks are capable of learning very complex functions
using the FORCE algorithm, the optimization requires a large number of learnings.
Therefore, the networks have to be restricted to being quite small ($\approx 100$ neurons)
and as such are less powerful. 

To get a simple criterion to estimate whether it might be possible to learn a given function 
using FORCE with 100 neurons, we look at a single sinusoidal of frequency $\omega$ and 
determine the range of $\omega$ in which an (unoptimized) network (i.e. GAIN, 
FEEDBACK, p, Erd�s Renyi) is able to learn the function (see fig. \ref{}). 
 

\section{Results}

\section{Discussion}

\blindtext


\begin{thebibliography}{}

\bibitem{mcdermott}
  Y.-F.~Chen {\it et al.},
  %``Microwave Photon Counter Based on Josephson Junctions,''
  Phys.\ Rev.\ Lett.\  {\bf 107}, 217401 (2011).
 

\end{thebibliography}

\end{document}

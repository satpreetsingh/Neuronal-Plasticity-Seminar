\documentclass[letter,scriptaddress,twocolumn, prl,showkeys]{revtex4-1}

    \usepackage{microtype}
    \usepackage{blindtext}

	\usepackage{amsmath}%,amssymb} 
	\usepackage{makeidx}
	\usepackage{amsfonts}
	\usepackage[ansinew]{inputenc}
	\usepackage[usenames,dvipsnames]{pstricks}
	\usepackage{subfigure}
	\usepackage{epsfig}
%	\usepackage{pst-grad} % For gradients
	\usepackage{pst-plot} % For axes
	\usepackage[colorlinks,hyperindex]{hyperref}
	\hypersetup
	{
		colorlinks,%
		citecolor=black,%
		linkcolor=black,%
		urlcolor=black,%
	}

%--- Theorem like environments ----
	\newtheorem{theorem}{Theorem}
	\newtheorem{corolary}{Corolary}
	\newtheorem{prop}{Proposition}
	\newtheorem{definition}{Definition}
	\newtheorem{example}{Example}
	\newtheorem{exercise}{Exercise}
	\newtheorem{lemma}{Lemma}

%--- Definindo algumas frescuras.
%\numberwithin{equation}{subsection}
	%\numberwithin{equation}{section}

    \newcommand{\ix}[1]{_\mathrm{#1}}
    \newcommand{\Ix}[1]{^\mathrm{#1}}

	\setlength\textheight{24.5cm}
	
\makeindex

%--------------------------------------------------------
\begin{document}

\title{Optimizing Structural Properties Of Neural Networks With Genetic Algorithms}

\author{T. Staudt, E. Schultheis}
\email{thomas.staudt@stud.uni-goettingen.de, erik.schultheis@uni-goettingen.de}
\affiliation{University of G�ttingen}

\date{today}

\begin{abstract}
    In this article we examine how structural properties of neural
    networks, like the size or local synapse densities, affect their
    learning success for various tasks. To do so, we look at a rate based
    model for neural networks and apply the FORCE rule for the learning
    process. A sophisticated matching algorithm
    allows us to quantify the learning success of a network so that the
    fitness of structural characteristics, expressed by integer or
    floating-point parameters, can be evaluated. We then use evolutionary
    optimization methods on these parameters to show that (1) ring
    topologies are generally more successful than strictly random
    topologies, (2) *FAT
    LIST OF INCREDIBLE RESULTS THAT CHANGE THE UNIVERSE*
\end{abstract}

\keywords{genetic algorithms, evolutionary algorithms, FORCE learning, rate networks}

\maketitle

\section{Introduction}
Looking at the field of artificial neuronal networks, one finds a great
variety of different models describing both the dynamic of networks as well
as their plasticities, i.e.\ their learning behavior.
But what exactly makes artificial neuronal networks learn successfully? And
why are certain types of networks and certain learning rules effective for
some problems, but fail at solving others? Confronting and eventually
answering these questions is critical when trying to gain a deep
understanding of neural networks and artificial intelligence. 

Using the above questions as a guideline, we decided to analyze what
network characteristics makes the FORCE algorithm, introduced in
\citep{FORCE}, especially suitable for learning periodic patterns.  More
precisely, we wanted to identify which properties of a network are
important for its learning success, and how to choose these properties'
values for an optimal result.  To this end, one option would be to take an
initial network and an adequate periodic pattern to be learned, and to
optimize the network's internals for this very task. Then patterns may
emerge that hint at how an optimal network design looks like. 

However, we took a different approach:
Instead of fine tuning single internal weights, we were concerned with the
general rules of the network's structure. So we wanted our networks to
be random to a certain degree, which seems biologically plausible, but we
also wanted them to follow certain patterns and allow for specialization:
How dense are the synapses distributed? Is there a strong compartmentalization?
Are recurrent structures frequent? How strong should the feedback signal
be, and how strong should the synapses fire on average?

To address these issues we looked at \emph{parameterized random networks}, or
in the language of genetics, at different \emph{network genotypes}. These
genotypes contain parametric information about the networks to be created,
the \emph{phenotypes}. Different phenotypes with the same genotype may thus
have strong quantitative differences, but they share structural qualities
whose fitness for a preassigned task we wanted to analyze and improve.

Since calculating the fitness of a genotype required the evaluation of many
different phenotypes and because of the resulting stochastic nature of the
problems we faced, we decided to use genetic algorithms for the
optimization process. Other more direct optimization methods like simulated
annealing or gradient descent seemed unfeasible for this.


\section{Concepts and Methods}
\paragraph{Network Model} We chose the same rate based network architecture
that is used in the original FORCE publication \citep{FORCE} (denoted as
architecture A) and also in \citep{RM}, where our notation mainly stems
from. So we look at a network $\mathcal{N}$ with $N$ internal neurons and
states $x_i$ for $1 \le i \le N$ that loosely represent the neurons
membrane potential. The firing rate $r_i$ of the $i$-th internal neuron is
given by $r_i = \tanh x_i$.  Further, the network contains $R$ readout
neurons with states $z_i$ for $1 \le i \le R$, where
\begin{equation} 
    z_i = \sum_{j=1}^{N}\omega\Ix{read}_{ij}\,r_j~.
\end{equation}
Taking into account both internal dynamics as well as an external feedback
pathway, the dynamics $t \mapsto x_i(t)$ of the single neurons are governed
by \footnote{~Note that we did not include external input signals in this
formula, as is done in \citep{FORCE} and \citep{RM}. We in fact enabled
inputs in our code base, but did not use them for the main results.}
\begin{equation}
    \dot{x}_i(t) = -x_i(t) + \sum_{i=1}^N \omega\Ix{rec}_{ij}\, r_j(t) + 
                   \sum_{i=1}^R \omega\Ix{fb}_{ij}\, z_j(t)~,
\end{equation}
where we introduce the internal recurrent synapse weights
$\omega\Ix{rec}_{ij}$ and the feedback synapse weights
$\omega\Ix{fb}_{ij}$. One can see that the neurons exponentially
decay if they are not presented stimulus by other neurons. 
We solved this system of differential equations by
a simple Newton method with integration time steps $dt = 0.01$, which was
also used in \citep{FORCE}. (How we constructed $\omega\Ix{rec}$ and
$\omega\Ix{fb}$ will be discussed in \emph{Parametric Random Networks})

\paragraph{Learning} Learning took place by stepwisely adapting the weights
$\omega\Ix{read}$ so that the dynamics $t\mapsto r_i(t)$ of the readout
rates should eventually match a given periodic input pattern $t\mapsto
a_i(t)$ as accurately as possible. To update $\omega\Ix{read}$ we chose
a variant of the FORCE algorithm \citep{FORCE} that rapidly modifies the
feedback loop to keep the errors $|r_i(t) - a_i(t)|$ small from the
beginning on.

While the authors of \citep{FORCE} update the weights $\omega\Ix{read}$ in
fixed intervals (every second time step), we used an adaptive mechanism to
determine how often learning should occur. This was initially introduced
for performance reasons, as the single learning steps proved to be the main
computational bottleneck, but we also found that the long time network
dynamics tended to be slightly more accurate for adaptive learning steps.

\paragraph{Parametric Random Networks -- Genotypes} Since we are not
interested in single network properties but rather want to find out which
structural elements of networks are important for learning success, we
introduced \emph{parametric random networks}, or \emph{network genotypes}.
Such a genotype $G$ consists of a set $\Theta$ of parameters that determine
which concrete networks are created when $G$ is expressed in a single
phenotype. Technically $G$ may be understood as a stochastic network
generator that returns a network $\mathcal{N}=G(\omega)$ for every seed
value $\omega$.
The genotypes' parameters $\theta\in\Theta$ were either integer or floating
point values and can be classified as affecting either (1) the topology of
$\omega\Ix{rec}$ or (2) the weight values of $\omega\Ix{rec}$ and
$\omega\Ix{fb}$. Important examples for (1) are the networks size $N$, the
occupation probability $p$ (for random Erd�s-Renyi networks), the
neighborhood range $k$ (for ring topologies), or the number $l$ of layers
(for feed-forward networks). We furthermore introduced ratio parameters
that allowed us to interpolate between different topologies.  Parameters of
type (2) were the feedback strength \texttt{feedback} (scaling of
$\omega\Ix{fb}$) and the gain value \texttt{gain} (scaling of
$\omega\Ix{rec}$ by ).

\paragraph{Fitness} For a given genotype $G$ and a given challenge $C$, the
fitness $f_C(G)$ of $G$ for $C$ is taken to be the expected value of
the success rates of $G$'s phenotypes when solving the tasks of the
challenge $C$. In our simulations, $f_C(G)$ was of course only
approximated; this was usually done by taking 20 to 100 sample phenotypes
and sample tasks, where the exact number depended on the standard deviation
of the previously drawn samples' success rates.

\paragraph{Genetic Optimization}
Genetic Optimization:

- Hint / vague description of the codebase

\paragraph{Tasks and Challenges}
For the optimization to work properly, the problems presented to the networks 
have (for the most part) to solvable by FORCE learning. While it was shown \citep{}
that large recurrent neural networks are capable of learning very complex functions
using the FORCE algorithm, the optimization requires a large number of learnings.
Therefore, the networks have to be restricted to being quite small ($\approx 100$ neurons)
and as such are less powerful. 

To get a simple criterion to estimate whether it might be possible to learn a given function 
using FORCE with 100 neurons, we look at a single sinusoidal of frequency $\omega$ and 
determine the range of $\omega$ in which an (unoptimized) network (i.e. GAIN, 
FEEDBACK, p, Erd�s Renyi) is able to learn the function (see fig. \ref{}). 

Since we expect/hope that the optimized network extends this range of possible frequencies,
we choose the frequencies for the optimiztion tasks from the slighly larger interval 
$[0.6e-2, 4]$ such that their logarithms are uniformly spaced. This allows for good resolution 
for both high and low frequencies.

\paragraph{Assessing the quality of reconstructed functions}
In order to optimize the function reconstruction of a force network, a measure $f(g, g^\prime)$ for
the concordance of target $g$ and network output $g^\prime$ is necessary. Simply using the difference
$f_d(g, g^\prime) = \left\| g-g^\prime \right\|_2$ has the drawback that small phase differences could 
produce huge differences in quality, even though $g^\prime$ has the exact same shape as $g$. 

A measure that is timeshift independent and maximized for functions of identical shape is the 
maximum of the cross correlation 
\begin{align}
f_c(g, g^\prime) = \max_t \left<g, g^\prime(\cdot - t)\right> / \sqrt{\left< g, g\right> \left<g^\prime, g^\prime \right>}.
\end{align}
Algorithmically, the signals are split into overlapping chunks which are correlated separately. 
This saves computational power and prevents minute differences in frequency to accumulate into 
a huge phase shift, i.e. the phase only has to remain relatively
constant over the timescale of a single chunk (i.e. about 10 seconds of simulated time).

This measure becomes problematic for functions that are very close to zero for a longer time 
(order of chunk size), but since we are going to work with superpositions of sinusoidals of 
different frequencies, this drawback is unproblematic.

 

\section{Results}

\section{Discussion}

\blindtext



\bibliography{sources}

\end{document}
